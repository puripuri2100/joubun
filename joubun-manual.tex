\documentclass{ltjsarticle}

\usepackage{joubun}

\begin{document}

\begin{mokuji}
\hen{債権}
\syou{総則}
\setu{多数当事者の債権及び債務}
\kan{保証債務}
\moku{賃金等根保証契約}
\hen{親族}
\husoku
\end{mokuji}

\hen{債権}
\syou{総則}
\setu{多数当事者の債権及び債務}
\kan{保証債務}
\moku{賃金等根保証契約}

\jou[(共同保証人間の求償権)]{第四百四十二条から第四百四十四条までの規定は、数人の保証人がある場合において、そのうちの一人の保証人が、主たる債務が不可分であるため又は各保証人が全額を弁済すべき旨の特約があるため、その全額又は自己の負担部分を超える額を弁済したときについて準用する。}
\kou{第四百六十二条の規定は、前項に規定する場合を除き、互いに連帯しない保証人の一人が全額又は自己の負担部分を超える額を弁済したときについて準用する。}


\edajou[(貸金等根保証契約の元本の確定事由)]{次に掲げる場合には、貸金等根保証契約における主たる債務の元本は、確定する。}
\gou{債権者が、主たる債務者又は保証人の財産について、金銭の支払を目的とする債権についての強制執行又は担保権の実行を申し立てたとき。ただし、強制執行又は担保権の実行の手続の開始があったときに限る。}
\sakujo{gou}
\gou{主たる債務者又は保証人が死亡したとき。}

\sakujo{jou}

\jou{債権は、譲り渡すことができる。ただし、その性質がこれを許さないときは、この限りでない。}

\hen{親族}

\begin{husoku}
\kou{削除}
\kou{この法律に定めるもののほか、この法律の施行に伴う経過措置及び関係法律の整備その他必要な事項については、別に法律で定める。}
\end{husoku}


%\begin{description}[leftmargin=3\zw,labelsep=1\zw,itemindent=0\zw,font=\rmfamily,nosep]
%\item[テストトト] どこで生れたかとんと見当がつかぬ。何でも薄暗いじめじめした所でニャーニャー泣いていた事だけは記憶している。吾輩はここで始めて人間というものを見た。
%\end{description}
\end{document}